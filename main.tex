
%% bare_jrnl.tex
%% V1.3
%% 2007/01/11
%% by Michael Shell
%% see http://www.michaelshell.org/
%% for current contact information.
%%
%% This is a skeleton file demonstrating the use of IEEEtran.cls
%% (requires IEEEtran.cls version 1.7 or later) with an IEEE journal paper.
%%
%% Support sites:
%% http://www.michaelshell.org/tex/ieeetran/
%% http://www.ctan.org/tex-archive/macros/latex/contrib/IEEEtran/
%% and
%% http://www.ieee.org/



% *** Authors should verify (and, if needed, correct) their LaTeX system  ***
% *** with the testflow diagnostic prior to trusting their LaTeX platform ***
% *** with production work. IEEE's font choices can trigger bugs that do  ***
% *** not appear when using other class files.                            ***
% The testflow support page is at:
% http://www.michaelshell.org/tex/testflow/


%%*************************************************************************
%% Legal Notice:
%% This code is offered as-is without any warranty either expressed or
%% implied; without even the implied warranty of MERCHANTABILITY or
%% FITNESS FOR A PARTICULAR PURPOSE! 
%% User assumes all risk.
%% In no event shall IEEE or any contributor to this code be liable for
%% any damages or losses, including, but not limited to, incidental,
%% consequential, or any other damages, resulting from the use or misuse
%% of any information contained here.
%%
%% All comments are the opinions of their respective authors and are not
%% necessarily endorsed by the IEEE.
%%
%% This work is distributed under the LaTeX Project Public License (LPPL)
%% ( http://www.latex-project.org/ ) version 1.3, and may be freely used,
%% distributed and modified. A copy of the LPPL, version 1.3, is included
%% in the base LaTeX documentation of all distributions of LaTeX released
%% 2003/12/01 or later.
%% Retain all contribution notices and credits.
%% ** Modified files should be clearly indicated as such, including  **
%% ** renaming them and changing author support contact information. **
%%
%% File list of work: IEEEtran.cls, IEEEtran_HOWTO.pdf, bare_adv.tex,
%%                    bare_conf.tex, bare_jrnl.tex, bare_jrnl_compsoc.tex
%%*************************************************************************

% Note that the a4paper option is mainly intended so that authors in
% countries using A4 can easily print to A4 and see how their papers will
% look in print - the typesetting of the document will not typically be
% affected with changes in paper size (but the bottom and side margins will).
% Use the testflow package mentioned above to verify correct handling of
% both paper sizes by the user's LaTeX system.
%
% Also note that the "draftcls" or "draftclsnofoot", not "draft", option
% should be used if it is desired that the figures are to be displayed in
% draft mode.
%
\documentclass[journal]{IEEEtran}
\usepackage{blindtext}
\usepackage{graphicx}
\usepackage{textgreek}
\usepackage[font=small,labelfont=bf]{caption}
\usepackage{dirtytalk}

\usepackage{titlesec}
% Some very useful LaTeX packages include:
% (uncomment the ones you want to load)


% *** MISC UTILITY PACKAGES ***
%
%\usepackage{ifpdf}
% Heiko Oberdiek's ifpdf.sty is very useful if you need conditional
% compilation based on whether the output is pdf or dvi.
% usage:
% \ifpdf
%   % pdf code
% \else
%   % dvi code
% \fi
% The latest version of ifpdf.sty can be obtained from:
% http://www.ctan.org/tex-archive/macros/latex/contrib/oberdiek/
% Also, note that IEEEtran.cls V1.7 and later provides a builtin
% \ifCLASSINFOpdf conditional that works the same way.
% When switching from latex to pdflatex and vice-versa, the compiler may
% have to be run twice to clear warning/error messages.






% *** CITATION PACKAGES ***
%
%\usepackage{cite}
% cite.sty was written by Donald Arseneau
% V1.6 and later of IEEEtran pre-defines the format of the cite.sty package
% \cite{} output to follow that of IEEE. Loading the cite package will
% result in citation numbers being automatically sorted and properly
% "compressed/ranged". e.g., [1], [9], [2], [7], [5], [6] without using
% cite.sty will become [1], [2], [5]--[7], [9] using cite.sty. cite.sty's
% \cite will automatically add leading space, if needed. Use cite.sty's
% noadjust option (cite.sty V3.8 and later) if you want to turn this off.
% cite.sty is already installed on most LaTeX systems. Be sure and use
% version 4.0 (2003-05-27) and later if using hyperref.sty. cite.sty does
% not currently provide for hyperlinked citations.
% The latest version can be obtained at:
% http://www.ctan.org/tex-archive/macros/latex/contrib/cite/
% The documentation is contained in the cite.sty file itself.






% *** GRAPHICS RELATED PACKAGES ***
%
\ifCLASSINFOpdf
  % \usepackage[pdftex]{graphicx}
  % declare the path(s) where your graphic files are
  % \graphicspath{{../pdf/}{../jpeg/}}
  % and their extensions so you won't have to specify these with
  % every instance of \includegraphics
  % \DeclareGraphicsExtensions{.pdf,.jpeg,.png}
\else
  % or other class option (dvipsone, dvipdf, if not using dvips). graphicx
  % will default to the driver specified in the system graphics.cfg if no
  % driver is specified.
  % \usepackage[dvips]{graphicx}
  % declare the path(s) where your graphic files are
  % \graphicspath{{../eps/}}
  % and their extensions so you won't have to specify these with
  % every instance of \includegraphics
  % \DeclareGraphicsExtensions{.eps}
\fi
% graphicx was written by David Carlisle and Sebastian Rahtz. It is
% required if you want graphics, photos, etc. graphicx.sty is already
% installed on most LaTeX systems. The latest version and documentation can
% be obtained at: 
% http://www.ctan.org/tex-archive/macros/latex/required/graphics/
% Another good source of documentation is "Using Imported Graphics in
% LaTeX2e" by Keith Reckdahl which can be found as epslatex.ps or
% epslatex.pdf at: http://www.ctan.org/tex-archive/info/
%
% latex, and pdflatex in dvi mode, support graphics in encapsulated
% postscript (.eps) format. pdflatex in pdf mode supports graphics
% in .pdf, .jpeg, .png and .mps (metapost) formats. Users should ensure
% that all non-photo figures use a vector format (.eps, .pdf, .mps) and
% not a bitmapped formats (.jpeg, .png). IEEE frowns on bitmapped formats
% which can result in "jaggedy"/blurry rendering of lines and letters as
% well as large increases in file sizes.
%
% You can find documentation about the pdfTeX application at:
% http://www.tug.org/applications/pdftex





% *** MATH PACKAGES ***
%
%\usepackage[cmex10]{amsmath}
% A popular package from the American Mathematical Society that provides
% many useful and powerful commands for dealing with mathematics. If using
% it, be sure to load this package with the cmex10 option to ensure that
% only type 1 fonts will utilized at all point sizes. Without this option,
% it is possible that some math symbols, particularly those within
% footnotes, will be rendered in bitmap form which will result in a
% document that can not be IEEE Xplore compliant!
%
% Also, note that the amsmath package sets \interdisplaylinepenalty to 10000
% thus preventing page breaks from occurring within multiline equations. Use:
%\interdisplaylinepenalty=2500
% after loading amsmath to restore such page breaks as IEEEtran.cls normally
% does. amsmath.sty is already installed on most LaTeX systems. The latest
% version and documentation can be obtained at:
% http://www.ctan.org/tex-archive/macros/latex/required/amslatex/math/





% *** SPECIALIZED LIST PACKAGES ***
%
%\usepackage{algorithmic}
% algorithmic.sty was written by Peter Williams and Rogerio Brito.
% This package provides an algorithmic environment fo describing algorithms.
% You can use the algorithmic environment in-text or within a figure
% environment to provide for a floating algorithm. Do NOT use the algorithm
% floating environment provided by algorithm.sty (by the same authors) or
% algorithm2e.sty (by Christophe Fiorio) as IEEE does not use dedicated
% algorithm float types and packages that provide these will not provide
% correct IEEE style captions. The latest version and documentation of
% algorithmic.sty can be obtained at:
% http://www.ctan.org/tex-archive/macros/latex/contrib/algorithms/
% There is also a support site at:
% http://algorithms.berlios.de/index.html
% Also of interest may be the (relatively newer and more customizable)
% algorithmicx.sty package by Szasz Janos:
% http://www.ctan.org/tex-archive/macros/latex/contrib/algorithmicx/




% *** ALIGNMENT PACKAGES ***
%
%\usepackage{array}
% Frank Mittelbach's and David Carlisle's array.sty patches and improves
% the standard LaTeX2e array and tabular environments to provide better
% appearance and additional user controls. As the default LaTeX2e table
% generation code is lacking to the point of almost being broken with
% respect to the quality of the end results, all users are strongly
% advised to use an enhanced (at the very least that provided by array.sty)
% set of table tools. array.sty is already installed on most systems. The
% latest version and documentation can be obtained at:
% http://www.ctan.org/tex-archive/macros/latex/required/tools/


%\usepackage{mdwmath}
%\usepackage{mdwtab}
% Also highly recommended is Mark Wooding's extremely powerful MDW tools,
% especially mdwmath.sty and mdwtab.sty which are used to format equations
% and tables, respectively. The MDWtools set is already installed on most
% LaTeX systems. The lastest version and documentation is available at:
% http://www.ctan.org/tex-archive/macros/latex/contrib/mdwtools/


% IEEEtran contains the IEEEeqnarray family of commands that can be used to
% generate multiline equations as well as matrices, tables, etc., of high
% quality.


%\usepackage{eqparbox}
% Also of notable interest is Scott Pakin's eqparbox package for creating
% (automatically sized) equal width boxes - aka "natural width parboxes".
% Available at:
% http://www.ctan.org/tex-archive/macros/latex/contrib/eqparbox/





% *** SUBFIGURE PACKAGES ***
%\usepackage[tight,footnotesize]{subfigure}
% subfigure.sty was written by Steven Douglas Cochran. This package makes it
% easy to put subfigures in your figures. e.g., "Figure 1a and 1b". For IEEE
% work, it is a good idea to load it with the tight package option to reduce
% the amount of white space around the subfigures. subfigure.sty is already
% installed on most LaTeX systems. The latest version and documentation can
% be obtained at:
% http://www.ctan.org/tex-archive/obsolete/macros/latex/contrib/subfigure/
% subfigure.sty has been superceeded by subfig.sty.



%\usepackage[caption=false]{caption}
%\usepackage[font=footnotesize]{subfig}
% subfig.sty, also written by Steven Douglas Cochran, is the modern
% replacement for subfigure.sty. However, subfig.sty requires and
% automatically loads Axel Sommerfeldt's caption.sty which will override
% IEEEtran.cls handling of captions and this will result in nonIEEE style
% figure/table captions. To prevent this problem, be sure and preload
% caption.sty with its "caption=false" package option. This is will preserve
% IEEEtran.cls handing of captions. Version 1.3 (2005/06/28) and later 
% (recommended due to many improvements over 1.2) of subfig.sty supports
% the caption=false option directly:
%\usepackage[caption=false,font=footnotesize]{subfig}
%
% The latest version and documentation can be obtained at:
% http://www.ctan.org/tex-archive/macros/latex/contrib/subfig/
% The latest version and documentation of caption.sty can be obtained at:
% http://www.ctan.org/tex-archive/macros/latex/contrib/caption/




% *** FLOAT PACKAGES ***
%
%\usepackage{fixltx2e}
% fixltx2e, the successor to the earlier fix2col.sty, was written by
% Frank Mittelbach and David Carlisle. This package corrects a few problems
% in the LaTeX2e kernel, the most notable of which is that in current
% LaTeX2e releases, the ordering of single and double column floats is not
% guaranteed to be preserved. Thus, an unpatched LaTeX2e can allow a
% single column figure to be placed prior to an earlier double column
% figure. The latest version and documentation can be found at:
% http://www.ctan.org/tex-archive/macros/latex/base/



%\usepackage{stfloats}
% stfloats.sty was written by Sigitas Tolusis. This package gives LaTeX2e
% the ability to do double column floats at the bottom of the page as well
% as the top. (e.g., "\begin{figure*}[!b]" is not normally possible in
% LaTeX2e). It also provides a command:
%\fnbelowfloat
% to enable the placement of footnotes below bottom floats (the standard
% LaTeX2e kernel puts them above bottom floats). This is an invasive package
% which rewrites many portions of the LaTeX2e float routines. It may not work
% with other packages that modify the LaTeX2e float routines. The latest
% version and documentation can be obtained at:
% http://www.ctan.org/tex-archive/macros/latex/contrib/sttools/
% Documentation is contained in the stfloats.sty comments as well as in the
% presfull.pdf file. Do not use the stfloats baselinefloat ability as IEEE
% does not allow \baselineskip to stretch. Authors submitting work to the
% IEEE should note that IEEE rarely uses double column equations and
% that authors should try to avoid such use. Do not be tempted to use the
% cuted.sty or midfloat.sty packages (also by Sigitas Tolusis) as IEEE does
% not format its papers in such ways.


%\ifCLASSOPTIONcaptionsoff
%  \usepackage[nomarkers]{endfloat}
% \let\MYoriglatexcaption\caption
% \renewcommand{\caption}[2][\relax]{\MYoriglatexcaption[#2]{#2}}
%\fi
% endfloat.sty was written by James Darrell McCauley and Jeff Goldberg.
% This package may be useful when used in conjunction with IEEEtran.cls'
% captionsoff option. Some IEEE journals/societies require that submissions
% have lists of figures/tables at the end of the paper and that
% figures/tables without any captions are placed on a page by themselves at
% the end of the document. If needed, the draftcls IEEEtran class option or
% \CLASSINPUTbaselinestretch interface can be used to increase the line
% spacing as well. Be sure and use the nomarkers option of endfloat to
% prevent endfloat from "marking" where the figures would have been placed
% in the text. The two hack lines of code above are a slight modification of
% that suggested by in the endfloat docs (section 8.3.1) to ensure that
% the full captions always appear in the list of figures/tables - even if
% the user used the short optional argument of \caption[]{}.
% IEEE papers do not typically make use of \caption[]'s optional argument,
% so this should not be an issue. A similar trick can be used to disable
% captions of packages such as subfig.sty that lack options to turn off
% the subcaptions:
% For subfig.sty:
% \let\MYorigsubfloat\subfloat
% \renewcommand{\subfloat}[2][\relax]{\MYorigsubfloat[]{#2}}
% For subfigure.sty:
% \let\MYorigsubfigure\subfigure
% \renewcommand{\subfigure}[2][\relax]{\MYorigsubfigure[]{#2}}
% However, the above trick will not work if both optional arguments of
% the \subfloat/subfig command are used. Furthermore, there needs to be a
% description of each subfigure *somewhere* and endfloat does not add
% subfigure captions to its list of figures. Thus, the best approach is to
% avoid the use of subfigure captions (many IEEE journals avoid them anyway)
% and instead reference/explain all the subfigures within the main caption.
% The latest version of endfloat.sty and its documentation can obtained at:
% http://www.ctan.org/tex-archive/macros/latex/contrib/endfloat/
%
% The IEEEtran \ifCLASSOPTIONcaptionsoff conditional can also be used
% later in the document, say, to conditionally put the References on a 
% page by themselves.





% *** PDF, URL AND HYPERLINK PACKAGES ***
%
%\usepackage{url}
% url.sty was written by Donald Arseneau. It provides better support for
% handling and breaking URLs. url.sty is already installed on most LaTeX
% systems. The latest version can be obtained at:
% http://www.ctan.org/tex-archive/macros/latex/contrib/misc/
% Read the url.sty source comments for usage information. Basically,
% \url{my_url_here}.





% *** Do not adjust lengths that control margins, column widths, etc. ***
% *** Do not use packages that alter fonts (such as pslatex).         ***
% There should be no need to do such things with IEEEtran.cls V1.6 and later.
% (Unless specifically asked to do so by the journal or conference you plan
% to submit to, of course. )


% correct bad hyphenation here
\hyphenation{op-tical net-works semi-conduc-tor}

\titleformat{\section}
  {\normalfont\Large\bfseries}{\thesection}{1em}{}[{\titlerule[0.8pt]}]
\begin{document}
%
% paper title
% can use linebreaks \\ within to get better formatting as desired
\title{Music Recommendation based on Personality Traits}



% \titleformat{\section}
%   {\normalfont\Large\bfseries}{\thesection}{1em}{}[{\titlerule[0.8pt]}]
% %
%
% author names and IEEE memberships
% note positions of commas and nonbreaking spaces ( ~ ) LaTeX will not break
% a structure at a ~ so this keeps an author's name from being broken across
% two lines.
% use \thanks{} to gain access to the first footnote area
% a separate \thanks must be used for each paragraph as LaTeX2e's \thanks
% was not built to handle multiple paragraphs
%

\author {Abhishek Paudel\textsuperscript{1}, Brihat Ratna Bajracharya\textsuperscript{2}, Miran Ghimire\textsuperscript{3}, Nabin Bhattarai\textsuperscript{4}, Daya Sagar Baral\textsuperscript{5}


Department of Electronics and Computer Engineering, Pulchowk Campus,\\
Institute of Engineering, Tribhuvan University, Nepal
\\
\\  \textsuperscript{1} 070bct502@ioe.edu.np
\\  \textsuperscript{2} 070bct513@ioe.edu.np
\\  \textsuperscript{3} 070bct521@ioe.edu.np
\\  \textsuperscript{4} 070bct522@ioe.edu.np
\\  \textsuperscript{5} dsb@ioe.edu.np
}
        % stops a space
% \thanks{M. Shell is with the Department
% of Electrical and Computer Engineering, Georgia Institute of Technology, Atlanta,
% GA, 30332 USA e-mail: (see http://www.michaelshell.org/contact.html).}% <-this % stops a space
% \thanks{J. Doe and J. Doe are with Anonymous University.}% <-this % stops a space

% note the % following the last \IEEEmembership and also \thanks - 
% these prevent an unwanted space from occurring between the last author name
% and the end of the author line. i.e., if you had this:
% 
% \author{....lastname \thanks{...} \thanks{...} }
%                     ^------------^------------^----Do not want these spaces!
%
% a space would be appended to the last name and could cause every name on that
% line to be shifted left slightly. This is one of those "LaTeX things". For
% instance, "\textbf{A} \textbf{B}" will typeset as "A B" not "AB". To get
% "AB" then you have to do: "\textbf{A}\textbf{B}"
% \thanks is no different in this regard, so shield the last } of each \thanks
% that ends a line with a % and do not let a space in before the next \thanks.
% Spaces after \IEEEmembership other than the last one are OK (and needed) as
% you are supposed to have spaces between the names. For what it is worth,
% this is a minor point as most people would not even notice if the said evil
% space somehow managed to creep in.



% The paper headers
% \markboth{Journal of \LaTeX\ Class Files,~Vol.~6, No.~1, January~2007}%
% {Shell \MakeLowercase{\textit{et al.}}: Bare Demo of IEEEtran.cls for Journals}
% The only time the second header will appear is for the odd numbered pages
% after the title page when using the twoside option.
% 
% *** Note that you probably will NOT want to include the author's ***
% *** name in the headers of peer review papers.                   ***
% You can use \ifCLASSOPTIONpeerreview for conditional compilation here if
% you desire.




% If you want to put a publisher's ID mark on the page you can do it like
% this:
%\IEEEpubid{0000--0000/00\$00.00~\copyright~2007 IEEE}
% Remember, if you use this you must call \IEEEpubidadjcol in the second
% column for its text to clear the IEEEpubid mark.



% use for special paper notices
%\IEEEspecialpapernotice{(Invited Paper)}




% make the title area
\maketitle


\begin{abstract}
%\boldmath

Music is an integral part of our life. People listen to music everyday as per their taste and mood. With the advancement and increase in volume of digital content, the choice for people to listen to diverse type of music has also increased significantly. Thus, the necessity of delivering the most suited music to the listeners has been an interesting field of research in computer science. One of the important measures to deliver the best music to listeners could be their personality traits. In order to determine the personality traits of a person, social media like Facebook can be a useful platform where people express their views on different matters, share their opinions and thoughts. This paper first describes the use of Naive Bayes classifier to determine the standard Big Five Personality Traits of a person based on their status updates on Facebook profile using basic natural language processing techniques, and then proceeds to present the use of thus obtained information about personality traits to enhance the widely implemented user-to-user collaborative filtering techniques for music recommendation.

\end{abstract}
% IEEEtran.cls defaults to using nonbold math in the Abstract.
% This preserves the distinction between vectors and scalars. However,
% if the journal you are submitting to favors bold math in the abstract,
% then you can use LaTeX's standard command \boldmath at the very start
% of the abstract to achieve this. Many IEEE journals frown on math
% in the abstract anyway.

% Note that keywords are not normally used for peerreview papers.
\begin{IEEEkeywords}
Recommender System, Collaborative Filtering, Personality Traits, Naive Bayes, Music
\end{IEEEkeywords}






% For peer review papers, you can put extra information on the cover
% page as needed:
% \ifCLASSOPTIONpeerreview
% \begin{center} \bfseries EDICS Category: 3-BBND \end{center}
% \fi
%
% For peerreview papers, this IEEEtran command inserts a page break and
% creates the second title. It will be ignored for other modes.
\IEEEpeerreviewmaketitle



\section{Introduction}


\subsection{Background}
On the Internet, where the number of choices is overwhelming, there is need to filter, prioritize and efficiently deliver relevant information in order to alleviate the problem of information overload, which has created a potential problem to many Internet users. Recommender systems solve this problem by searching through large volume of dynamically generated information to provide users with personalized content and services. Besides, these days social networks have become widely used and popular medium for information dissemination as well as the facilitators of social interactions. User contribution and activities provide a valuable insight into individual behavior, experiences, opinions and interests. Considering that personality, which uniquely identifies each one of us, affects a lot of aspects of human behavior, mental process and affective reactions, there is an enormous opportunity, for adding new personality based qualities in order to enhance the current collaborative filtering recommendation engine.\\
The Big Five Model or Five Factor Model of personality dimensions has emerged as one of the most well-researched and well-regarded measures of personality structure in recent years \cite{fivefactormodel}. The model five domains of personality: Openness, Conscientiousness, Extroversion, Agreeableness and Neuroticism, were conceived by Tupes and Christal \cite{tupes} as the fundamental traits that emerged from analyses of previous personality tests. McCrae, Costa and John \cite{mccrae} continued five factor model research and consistently found generality across age, gender and cultural lines.
The Big Five Model traits are characterized by the following:
\begin{enumerate}
\item Openness to Experience: Openness is a general appreciation of art, emotion, adventure, unusual ideas, imagination, curiosity, and variety of experience.
\item Conscientiousness: Conscientiousness is a tendency to display self-discipline, act dutifully and strive for achievement against measures or outside expectations.
\item Extraversion: Extraversion is characterized by breadth of activities, surgency from external activity/situations and energy creation from external means.
\item Agreeableness: The agreeableness trait reflects individual differences in general concern for social harmony.
\item Neuroticism: Neuroticism is the tendency to experience negative emotions, such as anger, anxiety or depression.
\end{enumerate}

\subsection{Literature Review}
The inception of recommender systems goes back to the 90's with introduction of applications that provided personalized advice for users about products or services they might be interested in \cite{resnick}.

In 2005, Gonzalez \cite{gonzalez} proposed a first model based on psychological aspects, he uses Emotional Intelligence to improve on-line course recommendations.

In 2008, Recommender System based on personality traits \cite{nunes} was published, experimenting on recommender system with the personality. The basically tired to recommend a person, in a voting scenario. Here recommendation was based on those psychological aspect of candidates and an imaginary person who they dreamed as ideal candidate. System used 30 facets of big 5 personality traits and only big 5 personality traits as the psychological measures of the users.

In 2014, Improving Music Recommender System. What can we learn from research on music tastes? \cite{laplante} was published which discuss about the music tastes from psychological point of view and uses psychology of music to identify the correlates of music tastes and to understand how music tastes are formed and evolve through time. It reveals the importance of social influences on music tastes and provides a basic suggestion for the design of music recommender system.

Also in 2014, Enhancing Music Recommender System with Personality Information andEmotional States \cite{bruce} was published, that researches to improve the music recommendation by including personality and emotional states. The proposal offers a great insight on how a recommendation engine can be improved with the personality via the series of steps.

In 2016, A Comparative Analysis of Personality Based Music Recommendation System \cite{melissa} was published which describes a preliminary study on considering information about the target user's personality in music recommendation system. It proposes a five different kind of models for the personality based music recommendation system.

This paper continues further with the experimentation of A Comparative Analysis of Personality Based Music Recommendation System whereby, the effects of personality based system on collaborative filtering has been studied rigorously.

\section{Methodology}

\subsection{Identification of Personality Traits}
\subsubsection{Data Collection}
In order to predict the personality in terms of Big Five Model, dataset for training the Naive Bayes classification model was obtained from myPersonality website\cite{dataset}. It consisted of collections of status updates of Facebook users along with their personality classification scores in terms of big five personality traits. This trained model would then use status updates from users' Facebook profile to determine their personality traits. Facebook Graph API has been implemented to collect status updates from user's Facebook profile.
\subsubsection{Data Preprocessing}
Data preprocessing included converting the status updates into vector representation with the use of bag-of-word model. The preprocessing tasks performed are depicted in figure 1.
\end{figure}
\begin{figure}[h!]\centering
\includegraphics[width=3in,height=5in,clip,keepaspectratio]{preprocessing.png}
\caption{Preprocessing tasks}
\label{fig:1}
\end{figure}

\subsubsection{Classifier Model}
Naive Bayes classifier was used for the classification of the status update text. It was of multinomial type since the frequency of occurrence of each feature in the feature vector is important and distribution of the feature is in discrete form.
In order to understand how Naive Bayes classifier \cite{naive} work, briefly understanding the concept of Bayes' rule is important.\\
Given the set of features $(x_1,x_2,x_3,\cdots, x_n)$, \\
Mathematically Bayes theorem can be written as:\\
\begin{equation}\label{eq:1}
P(C_{k}|x) = \frac{(P(C_{k}) * P(x|C_{k})} { P(x)}
\end{equation}
where,\\
$P(C_{k} |x)$ is the posterior probability of class 'c' given the attributes x \\
$P(C_{k})$ is the prior probability of class \\
$P(x|C_{k}$ is the likelihood which is the conditional probability of attributes being in the given class $C_k$.\\
$P(x)$ is called evidence \\
$k$ is used to denote the class label \\
Naive Bayes makes the independence assumption, so that \ref{eq:1} can be written as:\\
\begin{align}
\begin{split}
\begin{equation}\label{eq:naive}
P(C_{k}|x) = argmax \frac{(P(C_{k}) * P(x_1|C_{k}) * P(x_2|C_{k})*.....* (P(x_n|C_{k})} { P(x)} \\
\approx  (P(C_{k}) * P(x_1|C_{k}) * P(x_2|C_{k})*.....* (P(x_n|C_{k})
\end{equation}
\end{split}
\end{align}

which is the required equation of Naive Bayes used for the classification of text.

\paragraph{Additive Smoothing}

In statistics, additive smoothing \cite{additive}, also called Laplace smoothing is a technique used to smooth categorical data. Give an observation $x = (x_1,x_2,\cdots,x_d)$ from a multinomial distribution with N trials and parameter vector $\theta = (\theta_1,\theta_2,\cdots,\theta_d)$, a smoothed version of data given the estimator:\\
\begin{equation}\label{eq:smooth}
  \theta_i = \frac{x_i + \alpha}{N+ \alpha d}
\end{equation}
When $\alpha = 1 $ in \ref{eq:smooth}, it's called add one Laplace smoothing which has been used as the smoothing technique in this research in order to cancel out the effect of zero term by assigning them a small probability.

\paragraph{Underfitting}

Underfitting \cite{naive} in the Naive Bayes Classifier, can occur if the probabilities result from conditional and prior are very small, in this case in order to prevent the model from underfitting resulting from the multiplication of the very small terms, log can be used in \ref{eq:naive}, after which final equation becomes:

\begin{equation}
P(C_k|x) = \log p(C_k) + \sum_{i=1}^{k} \log(x|C_k)
\end{equation}
which is the final equation used in the research for the classification of user's status update texts into the personality traits.

\paragraph{Overfitting}

In order to reduce the overfitting and finding the best model for the classifier, $5^{th}$-fold cross validation, technique has been used. The major advantage of this method is that all observations are used for both training and testing and each observation is used for testing exactly once \cite{cross}.

\paragraph{Optimization}

Naive Bayes classifier, as seen in \ref{eq:naive}, classifies features set into a class via the multiplication of the prior and conditional probability which requires their computation each time the classifier tries to classify the feature into class. In order to solve this problem, conditional and prior probability is precomputed and stored in ``HashTable'' \cite{naive}, where the conditional probability of each feature set is stored, which can be easily be retrieved and used for the classification.

\subsubsection{Classifier Output}
The final output of the Naive Bayes classifier are the probabilities of the input status update text to fall under each of the five classes of personality traits.


\subsection{Recommendation Engine}
The main purpose of this research is to understand how personality impacts on the collaborative filtering (CF) model and compare it with some popular models. All together, 8 different recommendation models were created as shown in the figure 2.
\begin{figure}[!h]
\centering
\includegraphics[width=3.5in,height=5.25in,clip,keepaspectratio]{recommendation_models.png}
\caption{Recommendation models studied}
\label{fig:2}
\end{figure}

\subsubsection{Global Baseline Algorithm}
Global Baseline algorithm provides a mechanism to compute the unknown rating with baseline (i.e ``global effects'') estimates of corresponding users and items.
Mathematically,
Suppose $\mu$ be the system wide average rating, $b_x$ be the overall user rating deviation from system average and $b_i$ be the deviation in rating for a music $i$ then global base line algorithm rates a music $i$ for an user $x$ as:
\begin{equation}\label{eq:baseline}
  Global Baseline Estimate[R_{x,i}] = \mu + b_x + b_i
\end{equation}
\subsubsection{User to User collaborative filtering}

  \paragraph{User to Rating matrix computation:} User-rating matrix is computed with rating data of different users. For the purpose of the research, this data was generated manually.
  \paragraph{Normalization of the rating:} It is done in order to make the average rating of the system zeros so that the unknown values can be padded with zeros.
Mathematically,
Suppose $\mu_x$ be the average rating of the user x and $R_{x,i}$ represents a rating of user $x$ on music $i$ then normalized rating for an user $x$ on music $i$ can be computed as:
\begin{equation}\label{eq:normal}
  Normalized Rating[NR_{x,i}] = R_{x,i} - \mu
\end{equation}
\paragraph{Computing similar users:} In order to compute similar users, two metrics has been used: similarity based on the rating matrix of the user and similarity based on the personality. In both of the cases, the similar users are computed with the help of cosine similarity after the normalization of the rating.
Mathematically,
Suppose $r_a = [r_a1,r_a2,\cdots,r_an]$ be the user rating matrix of the user a and  $r_b = [r_b1,r_b2,\cdots,r_bn]$ be the user rating matrix of user b, then cosine similarity between user a and b can be obtained as:
\begin{equation}
  similarity_{a,b} = \frac{r_a1*r_b1 + r_a2*r_b2 +\cdots+ r_an*r_bn}{\sqrt{{r_a1}^2+{r_a2}^2+\cdots+{r_an}^2} * \sqrt{{r_b1}^2+{r_b2}^2+\cdots+{r_bn}^2} }
\end{equation}
Similarly, users with similar personality are computed with the help of personality vector.\hfill
\paragraph{Rating prediction:} A rating for user $x$ on music $i$ with the help of $N$ neighbor is computed by taking the weighted average rating of the neighbors.
\begin{equation}\label{eq:cf}
  r_{x,i} = \frac{\sum_{y=1}^N s_{x,y}*r_{y,i}}{\sum_{y=1}^N s_{x,y}}
\end{equation}
\paragraph{Recommendation:} After prediction of the rating, top-N items can be recommended to the users.
\end{itemize}
\subsubsection{Combination of Global Baseline and User to User collaborative filtering}
The equation \ref{eq:baseline} and \ref{eq:cf} can be combined as use together as:
\begin{equation}
  r_{x,i} = baseline_{x,i}+\frac{\sum_{y=1}^N s_{x,y}*(r_{y,i}-baseline_{y,i})}{\sum_{y=1}^N s_{x,y}}
\end{equation}
where,\\
$r_{x,i}$ is the rating on music $i$ by user $x$ \\
$baseline_{x,i}$ is the baseline estimate on music $i$ by user $x$ \\
$baseline_{y,i}$ is the baseline estimate on music $i$ by user $y$ \\
$s_{x,y}$ is the similarity between user $x$ and $y$ \\
$N$ is the total neighbors used for the recommendation
\subsubsection{Matrix Factorization}
Matrix factorization \cite{latent} involves in a factorization of a matrix to find out tow or more matrices such that when factors are multiplied together, original matrix in obtained. In recommender system, the matrix factorization is employed to predict the missing ratings such that the values would be consistent with the existing rating in the matrix. The intuition behind using matrix factorization, is that it is assumed there should be some latent features that determine how a user rates a music. For example two users would give high rating to a certain music if they both like the singer of the music or if the music is of same genre. Hence, if these latent features can be discovered, we should be able to predict a rating with respect to a certain user and a certain music, because the features associated with the user should match with the features associated with the music.

\section{Result and Analysis}
\subsection{Evaluation of Naive Bayes Model}

The followings tables show the confusion matrix of Naive Bayes for Big Five Personality classes:
\begin{table}[!ht]
\centering
\begin{tabular}{ |c|c|c| }
 \hline
 N =50 & Predicted:Yes & Predicted: No \\
 \hline
 Actual:Yes&3 & 12 \\
 \hline
 Actual:No&8 & 27 \\
 \hline
\end{tabular}
\caption{Confusion Matrix of Openness class}

\end{table}

\begin{table}[!ht]
\centering
\begin{tabular}{ |c|c|c| }
 \hline
 N =50 & Predicted:Yes & Predicted: No \\
 \hline
 Actual:Yes&9 & 15 \\
 \hline
 Actual:No&4 & 22 \\
 \hline
\end{tabular}
\caption{Confusion Matrix of Conscientiousness class}
\end{table}

\begin{table}[!ht]
\centering
\begin{tabular}{ |c|c|c| }
 \hline
 N =50 & Predicted:Yes & Predicted: No \\
 \hline
 Actual:Yes&20 & 11 \\
 \hline
 Actual:No&12 & 7 \\
 \hline
\end{tabular}
 \caption{Confusion Matrix of Extraversion class}
\end{table}

\begin{table}[!ht]
\centering
\begin{tabular}{ |c|c|c| }
 \hline
 N =50 & Predicted:Yes & Predicted: No \\
 \hline
 Actual:Yes&12 & 11 \\
 \hline
 Actual:No&13 & 14 \\
 \hline
\end{tabular}
 \caption{Confusion Matrix of Agreeableness class}
\end{table}

\begin{table}[!ht]
\centering
\begin{tabular}{ |c|c|c| }
 \hline
 N =50 & Predicted:Yes & Predicted: No \\
 \hline
 Actual:Yes&20 & 10 \\
 \hline
 Actual:No&15 & 5 \\
 \hline
\end{tabular}
 \caption{Confusion Matrix of Neuroticism class}
\end{table}

The following table shows f-measure of the Naive Bayes model for Big Five Personality classes:
\begin{table}[!ht]
\centering
\begin{tabular}{ |c|c| }
 \hline
 Class & f-measure \\
 \hline
 Openness&0.00\\
 \hline
 Conscientiousness&0.00 \\
 \hline
 Extraversion&0.00 \\
 \hline
 Agreeableness&0.00 \\
 \hline
 Neuroticism&0.00 \\
 \hline
\end{tabular}
 \caption{f-measures for Big Five Personality classes}
\end{table}

\subsection{Evaluation of Recommendation System}

The following table shows RMSE of various recommendation models:\\
  \begin{table}[!ht]
    \centering
    \begin{tabular}{| l | c |}
      \hline
      {\bf Recommendation Model} & {\bf RMSE}\\
      \hline
      User to User Collaborative Filtering with User & \\ Rating Matrix with combination of Global Baseline & 4.72\\
      \hline
      User to User Collaborative Filtering with User & \\ Rating Matrix & 3.89\\
      \hline
      User to User Collaborative Filtering with User & \\ Personality Matrix & 3.20\\
      \hline
      User to User Collaborative Filtering with Weighted & \\ Average of User Personality Matrix and User Rating Matrix & 3.2\\
      \hline
      User to User Collaborative Filtering with User & \\ Personality Matrix with combination of Global Baseline & 3.10\\
      \hline
      User to User Collaborative Filtering with Weighted & \\ Average of User Personality Matrix and User rating Matrix & \\ with combination of Global Baseline Algorithm & 3.04\\
      \hline
      Global Baseline Algorithm & 2.86\\
      \hline
      Matrix Factorization & 0.88\\
      \hline
    \end{tabular}
    \caption{RMSE of Recommendation System Models}
  \end{table}

The following figures show effects of change in number of nearest neighborhood in the different collaborative filtering models.\\

\begin{figure}[!ht]
  \centering
    \includegraphics[width=3.5in,height=5.25in,clip,keepaspectratio]{rmse_cf.png}
    \caption{RMSE of Collaborative Filtering with User Rating Matrix}
\end{figure}\\

\begin{figure}[!ht]
  \centering
    \includegraphics[width=3.5in,height=5.25in,clip,keepaspectratio]{rmse_cf_personality.png}
    \caption{RMSE of Collaborative Filtering with similarity in terms of Personality Matrix}
\end{figure}\\

\begin{figure}[!h]
  \centering
    \includegraphics[width=3.5in,height=5.25in,clip,keepaspectratio]{rmse_cf_global.png}
    \caption{RMSE of Collaborative Filtering combined with Global Baseline with User Rating Matrix}
\end{figure}\\

\begin{figure}[!h]
  \centering
    \includegraphics[width=3.5in,height=5.25in,clip,keepaspectratio]{rmse_cf_global_personality.png}
    \caption{RMSE of Collaborative Filtering combined with Global Baseline with User Personality Matrix}
\end{figure}\\

\begin{figure}[!h]
  \centering
    \includegraphics[width=3.5in,height=5.25in,clip,keepaspectratio]{rmse_cf_average.png}
    \caption{RMSE of Collaborative Filtering with User Rating and Personality Matrix}
\end{figure}\\

\begin{figure}[!h]
  \centering
    \includegraphics[width=3.5in,height=5.25in,clip,keepaspectratio]{rmse_cf_combined_average.png}
    \caption{RMSE of Collaborative Filtering with User Rating and Personality Matrix combined with Global Baseline}
\end{figure}\\



\subsubsection{Latent Factor}
The following figure shows the RMSE of matrix factorization when number of iterations is varied:

\begin{figure}[!ht]
\centering
\includegraphics[width=3.5in,height=5.25in,clip,keepaspectratio]{rmse_step.png}
\caption{RMSE of Matrix Factorization vs Number of Iterations}
\label{fig:rmse_step}
\end{figure}\\

The following figure shows the RMSE of matrix factorization when k is varied with number of iterations is fixed at 1000:
\begin{figure}[!ht]
\centering
\includegraphics[width=3.5in,height=5.25in,clip,keepaspectratio]{rmse_k.png}
\caption{RMSE of Matrix Factorization vs Number of Latent Factors}
\label{fig:rmse_k}
\end{figure}\\

Comparing the above models, it can be seen that the result of user to user collaborative filtering with the personality has slightly better result than the user to user collaborative filtering with the user Rating matrix but the matrix factorization outperforms them all. Besides, the result of weighted average of user similarity matrix with rating and personality also performs better than only a rating matrix but has a comparable result with the user to user collaborative filtering with personality to compute the similarity.

\section{Conclusion and Future Enhancement}
The paper presented the classification models that take Facebook user's status as input and classifies their personality based on big five personality traits. This information about personality traits is used by user to user collaborative filtering to find out similar users and recommend music to them. This recommendation model performs better than the user to user collaborative collaborative filtering with rating matrix but not as good as the matrix factorization. Besides, the recommendation model developed with personality has comparable result to weighted average of similarity using rating matrix and personality matrix. Hence, with reference to current scenario of the research, it can be concluded that personality traits of the user can be used to enhance existing user to user collaborative filtering that computes the similarity with the user rating matrix.
The future directions of this research could be focused towards the consideration of emojis in texts and some demographic information about the users in case of personality classification. Besides, the current recommendation engine as a whole suffers from cold start problem in case of music i.e. item ramp up problem. This can be solved with the content filtering method in order to create a profile of music. In addition, stability vs plasticity issue still prevails in user to user collaborative filtering with rating matrix and can be solved by giving low weights to the old rating of the users.

 


% needed in second column of first page if using \IEEEpubid
%\IEEEpubidadjcol

% An example of a floating figure using the graphicx package.
% Note that \label must occur AFTER (or within) \caption.
% For figures, \caption should occur after the \includegraphics.
% Note that IEEEtran v1.7 and later has special internal code that
% is designed to preserve the operation of \label within \caption
% even when the captionsoff option is in effect. However, because
% of issues like this, it may be the safest practice to put all your
% \label just after \caption rather than within \caption{}.
%
% Reminder: the "draftcls" or "draftclsnofoot", not "draft", class
% option should be used if it is desired that the figures are to be
% displayed while in draft mode.
%
%\begin{figure}[!t]
%\centering
%\includegraphics[width=2.5in]{myfigure}
% where an .eps filename suffix will be assumed under latex, 
% and a .pdf suffix will be assumed for pdflatex; or what has been declared
% via \DeclareGraphicsExtensions.
%\caption{Simulation Results}
%\label{fig_sim}
%\end{figure}

% Note that IEEE typically puts floats only at the top, even when this
% results in a large percentage of a column being occupied by floats.


% An example of a double column floating figure using two subfigures.
% (The subfig.sty package must be loaded for this to work.)
% The subfigure \label commands are set within each subfloat command, the
% \label for the overall figure must come after \caption.
% \hfil must be used as a separator to get equal spacing.
% The subfigure.sty package works much the same way, except \subfigure is
% used instead of \subfloat.
%
%\begin{figure*}[!t]
%\centerline{\subfloat[Case I]\includegraphics[width=2.5in]{subfigcase1}%
%\label{fig_first_case}}
%\hfil
%\subfloat[Case II]{\includegraphics[width=2.5in]{subfigcase2}%
%\label{fig_second_case}}}
%\caption{Simulation results}
%\label{fig_sim}
%\end{figure*}
%
% Note that often IEEE papers with subfigures do not employ subfigure
% captions (using the optional argument to \subfloat), but instead will
% reference/describe all of them (a), (b), etc., within the main caption.


% An example of a floating table. Note that, for IEEE style tables, the 
% \caption command should come BEFORE the table. Table text will default to
% \footnotesize as IEEE normally uses this smaller font for tables.
% The \label must come after \caption as always.
%
%\begin{table}[!t]
%% increase table row spacing, adjust to taste
%\renewcommand{\arraystretch}{1.3}
% if using array.sty, it might be a good idea to tweak the value of
% \extrarowheight as needed to properly center the text within the cells
%\caption{An Example of a Table}
%\label{table_example}
%\centering
%% Some packages, such as MDW tools, offer better commands for making tables
%% than the plain LaTeX2e tabular which is used here.
%\begin{tabular}{|c||c|}
%\hline
%One & Two\\
%\hline
%Three & Four\\
%\hline
%\end{tabular}
%\end{table}


% Note that IEEE does not put floats in the very first column - or typically
% anywhere on the first page for that matter. Also, in-text middle ("here")
% positioning is not used. Most IEEE journals use top floats exclusively.
% Note that, LaTeX2e, unlike IEEE journals, places footnotes above bottom
% floats. This can be corrected via the \fnbelowfloat command of the
% stfloats package.








% if have a single appendix:
%\appendix[Proof of the Zonklar Equations]
% or
%\appendix  % for no appendix heading
% do not use \section anymore after \appendix, only \section*
% is possibly needed

% use appendices with more than one appendix
% then use \section to start each appendix
% you must declare a \section before using any
% \subsection or using \label (\appendices by itself
% starts a section numbered zero.)
%


% use section* for acknowledgement
\section*{Acknowledgment}
We would like to express our sincere gratitude to the ​Department of Electronics and Computer Engineering at Pulchowk Campus, Institute of Engineering, Tribhuvan University for providing us the environment to work on the research.

We would also like to express our deepest sense of gratitude and thanks to our supervisor Mr. Daya Sagar Baral for providing invaluable insight and guidelines for this project.

We would also like to thank all of our friends who have directly and indirectly helped us in doing this research. Last but not the least, we place a deep sense of appreciation to our family members who have been constant source of inspiration for us.


% Can use something like this to put references on a page
% by themselves when using endfloat and the captionsoff option.
\ifCLASSOPTIONcaptionsoff
  \newpage
\fi



% trigger a \newpage just before the given reference
% number - used to balance the columns on the last page
% adjust value as needed - may need to be readjusted if
% the document is modified later
%\IEEEtriggeratref{8}
% The "triggered" command can be changed if desired:
%\IEEEtriggercmd{\enlargethispage{-5in}}

% references section

% can use a bibliography generated by BibTeX as a .bbl file
% BibTeX documentation can be easily obtained at:
% http://www.ctan.org/tex-archive/biblio/bibtex/contrib/doc/
% The IEEEtran BibTeX style support page is at:
% http://www.michaelshell.org/tex/ieeetran/bibtex/
%\bibliographystyle{IEEEtran}
% argument is your BibTeX string definitions and bibliography database(s)
%\bibliography{IEEEabrv,../bib/paper}
%
% <OR> manually copy in the resultant .bbl file
% set second argument of \begin to the number of references
% (used to reserve space for the reference number labels box)
\begin{thebibliography}{1}
\bibitem{dataset}
D. Stillwell and M. Kosinski. (2017, January). myPersonality DataSet. Retrieved from \url{ http://mypersonality.org/wiki/doku.php}

\bibitem{fivefactormodel}
Goldberg LR, et al. (2006). Five Factor Model of Personality:  \textit{The international personality item pool and the future of pulic-domain personality measures J Res Pers}, 40(1):8486.

\bibitem{tupes}
E. Tupes and R. Christal. (1992). Recurrent personality factors based on trait ratings. \textit{Journal of Personality}, 60(2): 225251.

\bibitem{mccrae}
R. McCrae and O. John. (1992). An introduction to the five-factor model and its applications. \textit{Journal of Personality}, 60(2): 175215.

\bibitem{laplante}
L. Audery. (2014). Improving music recommender systems: What can we learn from research on music tastes? \textit{ISMIR}.

\bibitem{bruce}
B. Ferwerda and S. Markus. (2014). Enhancing Music Recommender Systems with Personality Information and Emotional States: A Propoasal. \textit{UMAP Workshops}.

\bibitem{melissa}
O. Melissa, A. Micareli and G. Sansonetti. (2016). A Comparative Analysis of Personaility Based Music Recommender Systems. \textit{EMPIRE RecSys}

\bibitem{gonzalez}
G. Gonzalez and M. Miquel. (2017). Embedding Emotional Context in Recommendation System. \textit{20th International Florida Artifical Intelligence Research Society Conference-FLAIRS}.

\bibitem{resnick}
P. Resnick and H. R. Vairan. (1997). Recommender Systems. \textit{Communications of the ACM}, 40(3):56-58.

\bibitem{nunes}
Nunes, M.A.S.N. (2008). \textit{Recommender system based on personality traits}
\bibitem{kd}
G. Parsa. (2017, July). Document Classification. Retrieved from \url{http://www.kdnuggets.com/2015/01/text-analysis-101-document-classification.html}

\bibitem{naive}
Naive Bayes text classification. (2017, July). Retrieved from \url{https://nlp.stanford.edu/IR-book/html/htmledition/naive-bayes-text-classification-1.html}

\bibitem{demographic}
L. Safoury and A. Salah. (2013). Exploiting user demographic attributes for solving cold-start problem in recommender system.\textit{Lecture Notes on Software Engineering}, 1(3), 303
\bibitem{recommend}
A. Tejal. (2015). A Survey on Recommendation System. \textit{International Journal of Innovative Research in Advanced Engineering(IJIRA)}.
\bibitem{rmain}
F. O. Isinkaye, Y. O. Folajimi and B. A. Ojokoh. (2015). Recommendation Systems: Principles, methods and evaluation \textit{Egyptian Informatics Journal}.
\bibitem{latent}
Non-negative matrix factorization. (2017, July). Retrieved from \url{https://www.slideshare.net/BenjaminBengfort/non-negative-matrix-factorization}
\bibitem{manning}
C. Manning. (2017, June). Stanford NLP - Standford NLP Group. Retrieved from \url{https://nlp.standford.edu/manning}

\bibitem{traits}
Studying the big five personality traits-UK Essays. (2017. January). Retrieved from \url{https://ukessays.com/essays/psychology/studying-the-big-five-personality-traits.php}

\bibitem{eval}
D. Alexander. (2009). Collaborative Filtering and Recommender Systems

\bibitem{api}
Facebook Graph API. (2017, July). Retrieved from \url{https://developers.facebook.com/docs/graph-api}

\end{thebibliography}

% biography section
% 
% If you have an EPS/PDF photo (graphicx package needed) extra braces are
% needed around the contents of the optional argument to biography to prevent
% the LaTeX parser from getting confused when it sees the complicated
% \includegraphics command within an optional argument. (You could create
% your own custom macro containing the \includegraphics command to make things
% simpler here.)
%\begin{biography}[{\includegraphics[width=1in,height=1.25in,clip,keepaspectratio]{mshell}}]{Michael Shell}
% or if you just want to reserve a space for a photo:



% You can push biographies down or up by placing
% a \vfill before or after them. The appropriate
% use of \vfill depends on what kind of text is
% on the last page and whether or not the columns
% are being equalized.

%\vfill

% Can be used to pull up biographies so that the bottom of the last one
% is flush with the other column.
%\enlargethispage{-5in}



% that's all folks
\end{document}


